\documentclass[12pt,a4paper]{article}
\usepackage[utf8]{inputenc}
\usepackage{amsmath}
\usepackage{amsfonts}
\usepackage{amssymb}
\usepackage{makeidx}
\usepackage{graphicx}
\usepackage[left=2cm,right=2cm,top=2cm,bottom=2cm]{geometry}
\author{DIEGO HILDEBRANDO RAMIREZ AGUILERA,\\JONATHAN ALFEREZ TORRES,\\ALEJANDRO ALMARAZ QUINTERO,\\VICTOR HERNANDEZ VIDRIO,\\JUAN MANUEL NAVARRETE DIAZ}

\linebreak
\title{ROBOT TIPO ESFERICO}

\begin{document}
\maketitle
\linebreak
\linebreak
\linebreak


\includegraphics[width=450]{../Pictures/cvncbmbv.jpg} 
\clearpage
\section{introduccion}
En esta ocasion fue seleccionado un robot de tipo esferico donde se elaborara en un plazo de un cuatrimestre ademas de tener una idea clara del funcionamiento y de las caracteristicas que debe de llevar.

\subsection{Marco teorico}
Este tipo de configuración está compuesta por dos ejes rotacionales perpendiculares y uno lineal. Se denominan esféricos o polares porque sus ejes forman un sistema de coordenadas polar. Por medio de estas diversas articulaciones proporcionan al robot la capacidad para desplazar su brazo dentro de un espacio esférico.
Esta rota en su base, se inclina en su hombro, y cuenta con extensión y retracción en su brazo y su área de funcionamiento es una porción de esfera. Presenta algunos inconvenientes en el momento de realizar un simple movimiento de traslación o pérdida de precisión cuando este trabajar con cargas pesadas y con el brazo muy extendido
Este tipo de configuración se componen de 3 articulaciones o ejes: Dos ejes rotacionales que generan un movimiento rotativo. Un eje prismático que se encarga de realizar un movimiento lineal o deslizante. Las articulaciones rotativas son perpendiculares entre el primer y segundo segmento.
Generalmente se describe la posición del robot dando una descripción del marco de la herramienta, la cual está unida al órgano terminal, relativo al marco de la base, el cual está a su vez unido a la base fija del robot. El modelo cinemático directo es el problema geométrico que calcular la posición y orientación del efector final del robot. Dados una serie de ángulos entre las articulaciones, el problema cinemático directo calcula la posición y orientación del marco de referencia del efector final con respecto al marco de la base. 

\includegraphics[scale=0.5]{../Pictures/qsdqd.jpg} 
\section{obgetivos:}Realizar un robot de tipo esferico que tenga una aplicación util en la vida humana.
\linebreak
Cumplir con todo lo requerido para satisfacer las necesidades requeridas.
\linebreak

El robot tendra un uso para la selección de residuos como lo son las botellas pet y botellas de vidrio.
\linebreak

Levantar minimo 500Kg

\includegraphics[scale=0.5]{../Pictures/DCDSC.jpg} 
\includegraphics[scale=0.5]{../Pictures/DSVSDCX.jpg} 





\end{document}