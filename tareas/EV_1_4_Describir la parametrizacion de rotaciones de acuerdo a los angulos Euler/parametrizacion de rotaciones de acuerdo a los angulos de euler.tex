\documentclass[12pt,a4paper]{article}
\usepackage[utf8]{inputenc}
\usepackage{amsmath}
\usepackage{amsfonts}
\usepackage{amssymb}
\usepackage{makeidx}
\usepackage{graphicx}
\usepackage[left=2cm,right=2cm,top=2cm,bottom=2cm]{geometry}
\author{DIEGO HILDEBRANDO RAMIREZ AGUILERA} 
\title{parametrizacion de rotaciones de acuerdo a los angulos de euler}
\begin{document}
\maketitle
\section{Ángulos de Euler}
\subsection{Los ángulos de Euler constituyen un conjunto de tres coordenadas angulares que sirven para especificar la orientación de un sistema de referencia de ejes ortogonales, normalmente móvil, respecto a otro sistema de referencia de ejes ortogonales normalmente fijos.
Dados dos sistemas de coordenadas xyz y XYZ con origen común, es posible especificar la posición de un sistema en términos del otro usando tres ángulos:
La definición matemática es estática y se basa en escoger dos planos, uno en el sistema de referencia y otro en el triedro rotado. En el esquema adjunto serían los planos xy y XY. Escogiendo otros planos se obtendrían distintas convenciones alternativas, las cuales se llaman de Tait-Bryan cuando los planos de referencia son no-homogéneos (por ejemplo xy y XY son homogéneos, mientras xy y XZ no lo son).
}

\includegraphics[scale=1]{../../../../Pictures/260px-Eulerfch.jpg} 
\section{Rotaciones de Euler}
\subsection{Son los movimientos resultantes de variar uno de los ángulos de Euler dejando fijos los otros dos. Tienen nombres particulares:precesion,nutacion y rotacion intrinseca.
Este conjunto de rotaciones no es ni intrínseco ni extrínseco en su totalidad, sino que es una mezcla de ambos conceptos. La precesión es extrínseca, la rotación intrínseca lógicamente intrínseca, y la nutación es una rotación intermedia, alrededor de la línea de nodos.}

\includegraphics[scale=1]{../../../../Pictures/260px-Precession-nutation-ES.jpg} 

\section{Matriz de rotación}

\subsection{Para pasar directamente de la base 4, ligada al sólido, a la base 1, fija, podemos hacer sustituciones sucesivas, aunque los cálculos se vuelven rápidamente engorrosos. La matrix de rotación está formada por los productos escalares entre ambas bases}

\includegraphics[scale=1]{../../../../Pictures/e19c036b7219f7df0dce2f328c07d181.jpg} 


\end{document}