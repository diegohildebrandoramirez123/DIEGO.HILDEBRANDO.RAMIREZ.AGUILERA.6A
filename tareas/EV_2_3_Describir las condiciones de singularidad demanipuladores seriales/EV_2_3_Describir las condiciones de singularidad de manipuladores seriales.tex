\documentclass[12pt,a4paper]{article}
\usepackage[utf8]{inputenc}
\usepackage{amsmath}
\usepackage{amsfonts}
\usepackage{amssymb}
\usepackage{makeidx}
\usepackage{graphicx}
\usepackage[left=2cm,right=2cm,top=2cm,bottom=2cm]{geometry}
\author{DIEGO HILDEBRANDO RAMIREZ AGUILERA}
\title{condiciones de singularidad de manipuladores seriales}
\begin{document}
\maketitle
\section{Introduccion}
Un robot manipulador serie es una cadena cinematica abierta compuesta de una secuencia de elementos estructurales r´ıgidos, denominados eslabones, conectados entre s´ı a traves de articulaciones, que permiten el movimiento relativo de cada par
de eslabones consecutivos. Al final del ultimo eslabon puede anadirse una herramienta o dispositivo, denominado elemento terminal.Asimismo, a cada articulacion´ i se le asocia un sistema de coordenadas {i} que se utiliza paradescribir su posicion´y orientacion relativas. La relaci ´ on entre los sistemas de coordenadas asociados articulaciones consecutivas viene descrita
mediante una matriz de transformacion homogenea construida a partir de los parametros de Denavit-Hartenberg (D-H) (Craig,1989; Siciliano et al., 2008; Spong et al., 2006; 
\section{Descripcion}
La resolucion de la cinem ´ atica inversa consiste en hallar las ´
configuraciones asociadas a una pose concreta del elemento terminal. Como se comento en la introducci ´ on, las soluciones pue- ´
den agruparse en dos grupos: soluciones anal´ıticas, o en forma
cerrada, y soluciones numericas. ´
Para manipuladores no redundantes que poseen muneca es- ˜
ferica, (Pieper, 1968) demostr ´ o de forma constructiva que siem- ´
pre existe solucion anal ´ ´ıtica. Mas a ´ un, demostr ´ o tambi ´ en que ´
cualquier robot manipulador con tres articulaciones consecutivas cuyos ejes se cortan en un punto o son paralelos tiene solucion en forma cerrada. Los manipuladores no redundantes que ´
no poseen muneca esf ˜ erica y los manipuladores redundantes ´
no tienen solucion anal ´ ´ıtica en general, por lo que se emplean
metodos num ´ ericos para obtener al menos una buena aproxima- ´
cion de una de las soluciones. Estos m ´ etodos iterativos general- ´
mente se basan en el metodo de Newton para resolver sistemas ´
de ecuaciones no lineales (Buss, 2009; Buss and Kim, 2005).
No obstante, para el caso de robots manipuladores redundantes con muneca esf ˜ erica se puede hallar un procedimiento ´
para obtener el conjunto de soluciones en forma cerrada. Para
ello, si el manipulador tiene m GdL redundantes, se parametrizan m variables articulares. De esta forma se reduce el problema
al caso no redundante donde se pueden aplicar los metodos de ´
Pieper o (Paul, 1981) en donde se obtienen las soluciones en
funcion de las variables articulares parametrizadas. Para com- ´
pletar la solucion, se definen funciones que optimizan dichos ´
parametros en funci ´ on de un objetivo secundario, como evitar ´
singularidades, evitar limites articulares, etc.

\includegraphics[scale=2]{../../../Pictures/fvfvvf.jpg} 
\section{Analisis de singularidad}
En esta secci´on de presenta la metodolog´ıa
seguida en el an´alisis de singularidad. Con la
finalidad de abarcar todas las posibles configuraciones singulares del mecanismo espacial
tipo RRRCR, se har´a un an´alisis exhaustivo
de la ecuaci´on de velocidad, ver ecuaci´on (6),
utilizando toda la potencia matem´atica ofrecida
por los vectores y las matrices.
De esta manera, el conjunto de todas las posibilidades se obtiene intercambiando las diferentes
posiciones que pueden adoptar todos los t´erminos, tanto en el lado izquierdo (tres) como en
el lado derecho (un t´ermino) de la ecuaci´on (6).
As´ı pues, el total de posibilidades ser´ıa de 24. Sin
embargo, debido a que cada producto vectorial del
tipo (a×b)· c debe cumplir con el siguiente ciclo
de 6 identidades:


\end{document}